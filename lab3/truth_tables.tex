\documentclass[12pt,a4paper]{article}
\usepackage[utf8]{inputenc}
\usepackage[russian]{babel}
\usepackage{amsmath}
\usepackage{amsfonts}
\usepackage{amssymb}
\usepackage{geometry}
\usepackage{array}
\usepackage{booktabs}

\geometry{margin=2cm}

\title{Таблицы истинности для лабораторной работы 3}
\author{Анализ комбинационных и последовательных схем}
\date{\today}

\begin{document}

\maketitle

\section{Комбинационные схемы}

\subsection{Часть 1: Логическая функция $q = (a | b) \& (c | d)$}

\begin{table}[h]
\centering
\caption{Таблица истинности для части 1}
\begin{tabular}{|c|c|c|c|c|}
\hline
$a$ & $b$ & $c$ & $d$ & $q$ \\
\hline
0 & 0 & 0 & 0 & 0 \\
0 & 0 & 0 & 1 & 0 \\
0 & 0 & 1 & 0 & 0 \\
0 & 0 & 1 & 1 & 0 \\
0 & 1 & 0 & 0 & 0 \\
0 & 1 & 0 & 1 & 1 \\
0 & 1 & 1 & 0 & 1 \\
0 & 1 & 1 & 1 & 1 \\
1 & 0 & 0 & 0 & 0 \\
1 & 0 & 0 & 1 & 1 \\
1 & 0 & 1 & 0 & 1 \\
1 & 0 & 1 & 1 & 1 \\
1 & 1 & 0 & 0 & 0 \\
1 & 1 & 0 & 1 & 1 \\
1 & 1 & 1 & 0 & 1 \\
1 & 1 & 1 & 1 & 1 \\
\hline
\end{tabular}
\end{table}

\subsection{Часть 2: Логическая функция $q = b | c$}

\begin{table}[h]
\centering
\caption{Таблица истинности для части 2}
\begin{tabular}{|c|c|c|c|c|}
\hline
$a$ & $b$ & $c$ & $d$ & $q$ \\
\hline
X & 0 & 0 & X & 0 \\
X & 0 & 1 & X & 1 \\
X & 1 & 0 & X & 1 \\
X & 1 & 1 & X & 1 \\
\hline
\end{tabular}
\end{table}

\textbf{Примечание:} Выход $q$ зависит только от входов $b$ и $c$, входы $a$ и $d$ не влияют на результат.

\subsection{Часть 3: Мультиплексор 4x1}

\begin{table}[h]
\centering
\caption{Таблица истинности для части 3}
\begin{tabular}{|c|c|c|c|c|c|}
\hline
$a[3:0]$ & $b[3:0]$ & $c[3:0]$ & $d[3:0]$ & $e[3:0]$ & $q[3:0]$ \\
\hline
X & X & 0 & X & X & $b$ \\
X & X & 1 & X & X & $e$ \\
X & X & 2 & X & X & $a$ \\
X & X & 3 & X & X & $d$ \\
X & X & 4-15 & X & X & $F$ \\
\hline
\end{tabular}
\end{table}

\textbf{Примечание:} Мультиплексор выбирает один из четырех входов по управляющему сигналу $c$. При неопределенных значениях $c$ выход устанавливается в $F$.

\section{Последовательные схемы}

\subsection{Часть 4: Логика на основе тактового сигнала}

\begin{table}[h]
\centering
\caption{Логика работы модуля Part 4}
\begin{tabular}{|c|c|c|c|}
\hline
$clock$ & $a$ & $p$ & $q$ \\
\hline
0 & 0 & $p_{prev}$ & $q_{prev}$ \\
0 & 1 & $p_{prev}$ & $q_{prev}$ \\
1 & 0 & 0 & $p$ \\
1 & 1 & 1 & $p$ \\
\hline
\end{tabular}
\end{table}

\textbf{Примечание:} $p$ обновляется по положительному фронту $clock$, $q$ обновляется по отрицательному фронту $clock$.

\subsection{Часть 5: Счетчик с остановкой}

\begin{table}[h]
\centering
\caption{Логика работы счетчика Part 5}
\begin{tabular}{|c|c|c|c|}
\hline
$a$ & $q_{current}$ & $q_{next}$ & Описание \\
\hline
0 & 0-5 & $q + 1$ & Обычный счет \\
0 & 6 & 0 & Переполнение \\
1 & X & 4 & Принудительная остановка на 4 \\
\hline
\end{tabular}
\end{table}

\textbf{Примечание:} Счетчик считает от 0 до 6, при $a=1$ устанавливается значение 4.

\subsection{Часть 6: XOR логика с состоянием}

\begin{table}[h]
\centering
\caption{Логика работы модуля Part 6}
\begin{tabular}{|c|c|c|c|c|}
\hline
$a$ & $b$ & $state_{current}$ & $state_{next}$ & $q$ \\
\hline
0 & 0 & 0 & $a$ & $a \oplus b \oplus state$ \\
0 & 0 & 1 & $a$ & $a \oplus b \oplus state$ \\
0 & 1 & 0 & $state$ & $a \oplus b \oplus state$ \\
0 & 1 & 1 & $state$ & $a \oplus b \oplus state$ \\
1 & 0 & 0 & $state$ & $a \oplus b \oplus state$ \\
1 & 0 & 1 & $state$ & $a \oplus b \oplus state$ \\
1 & 1 & 0 & $a$ & $a \oplus b \oplus state$ \\
1 & 1 & 1 & $a$ & $a \oplus b \oplus state$ \\
\hline
\end{tabular}
\end{table}

\textbf{Примечание:} $state_{n+1} = (a == b) ? a : state$, $q = a \oplus b \oplus state$

\section{FSM модули}

\subsection{FSM Part 1: Счетчик 0-999}

\begin{table}[h]
\centering
\caption{Логика работы счетчика}
\begin{tabular}{|c|c|c|c|}
\hline
$reset$ & $q$ & $q_{n+1}$ & Описание \\
\hline
1 & X & 0 & Сброс счетчика \\
0 & 0-998 & $q + 1$ & Инкремент \\
0 & 999 & 0 & Переполнение \\
\hline
\end{tabular}
\end{table}

\subsection{FSM Part 2: Сдвиговый регистр-счетчик}

\begin{table}[h]
\centering
\caption{Логика работы сдвигового регистра}
\begin{tabular}{|c|c|c|c|c|}
\hline
$shift\_ena$ & $count\_ena$ & $data$ & $q_{n+1}$ & Описание \\
\hline
1 & 0 & X & $\{q[2:0], data\}$ & Сдвиг влево \\
0 & 1 & X & $q - 1$ (если $q \neq 0$) & Обратный счет \\
0 & 0 & X & $q$ & Сохранение \\
\hline
\end{tabular}
\end{table}

\subsection{FSM Part 3: Детектор паттерна 1101}

\begin{table}[h]
\centering
\caption{Состояния FSM детектора паттерна}
\begin{tabular}{|c|c|c|c|}
\hline
Состояние & Вход & Следующее состояние & $start\_shifting$ \\
\hline
S0 (IDLE) & 0 & S0 & 0 \\
S0 (IDLE) & 1 & S1 & 0 \\
S1 & 0 & S0 & 0 \\
S1 & 1 & S2 & 0 \\
S2 & 0 & S3 & 0 \\
S2 & 1 & S2 & 0 \\
S3 & 0 & S0 & 0 \\
S3 & 1 & S4 & 1 \\
S4 & 0 & S0 & 0 \\
S4 & 1 & S1 & 0 \\
\hline
\end{tabular}
\end{table}

\subsection{FSM Part 4: Продвинутый таймер}

\begin{table}[h]
\centering
\caption{Состояния FSM продвинутого таймера}
\begin{tabular}{|c|c|c|c|c|}
\hline
Состояние & Условие перехода & Следующее состояние & $counting$ & $done$ \\
\hline
S0-S3 & Паттерн 1101 найден & B0 & 0 & 0 \\
B0-B3 & 4 бита сдвинуты & COUNT & 0 & 0 \\
COUNT & $num\_cnt \geq target$ & WAIT & 1 & 0 \\
WAIT & $ack = 1$ & S0 & 0 & 1 \\
\hline
\end{tabular}
\end{table}

\textbf{Примечание:} $target\_count = (cnt + 1) \times 1000$, где $cnt$ - значение сдвигового регистра.

\section{Заключение}

Все таблицы истинности соответствуют реализованным модулям и подтверждают корректность их работы. Комбинационные схемы реализуют различные логические функции, а последовательные схемы демонстрируют работу с тактовыми сигналами и внутренними состояниями. FSM модули показывают сложную логику конечных автоматов с множественными состояниями и переходами.

\end{document}