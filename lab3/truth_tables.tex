\documentclass[12pt,a4paper]{article}
\usepackage[utf8]{inputenc}
\usepackage[russian]{babel}
\usepackage{amsmath}
\usepackage{amsfonts}
\usepackage{amssymb}
\usepackage{geometry}
\usepackage{array}
\usepackage{booktabs}

\geometry{margin=2cm}

\title{Таблицы истинности для лабораторной работы 3}
\author{Анализ комбинационных и последовательных схем}
\date{\today}

\begin{document}

\maketitle

\section{Комбинационные схемы}

\subsection{Часть 1: Логическая функция $y = c$}

\begin{table}[h]
\centering
\caption{Таблица истинности для части 1}
\begin{tabular}{|c|c|c|c|}
\hline
$a$ & $b$ & $c$ & $y$ \\
\hline
0 & 0 & 0 & 0 \\
0 & 0 & 1 & 1 \\
0 & 1 & 0 & 0 \\
0 & 1 & 1 & 1 \\
1 & 0 & 0 & 0 \\
1 & 0 & 1 & 1 \\
1 & 1 & 0 & 0 \\
1 & 1 & 1 & 1 \\
\hline
\end{tabular}
\end{table}

\subsection{Часть 2: Функция исключающего ИЛИ $y = a \oplus b \oplus c \oplus d$}

\begin{table}[h]
\centering
\caption{Таблица истинности для части 2}
\begin{tabular}{|c|c|c|c|c|}
\hline
$a$ & $b$ & $c$ & $d$ & $y$ \\
\hline
0 & 0 & 0 & 0 & 0 \\
0 & 0 & 0 & 1 & 1 \\
0 & 0 & 1 & 0 & 1 \\
0 & 0 & 1 & 1 & 0 \\
0 & 1 & 0 & 0 & 1 \\
0 & 1 & 0 & 1 & 0 \\
0 & 1 & 1 & 0 & 0 \\
0 & 1 & 1 & 1 & 1 \\
1 & 0 & 0 & 0 & 1 \\
1 & 0 & 0 & 1 & 0 \\
1 & 0 & 1 & 0 & 0 \\
1 & 0 & 1 & 1 & 1 \\
1 & 1 & 0 & 0 & 0 \\
1 & 1 & 0 & 1 & 1 \\
1 & 1 & 1 & 0 & 1 \\
1 & 1 & 1 & 1 & 0 \\
\hline
\end{tabular}
\end{table}

\subsection{Часть 3: Логическая функция $y = (a \cdot b) + (c \cdot d)$}

\begin{table}[h]
\centering
\caption{Таблица истинности для части 3}
\begin{tabular}{|c|c|c|c|c|}
\hline
$a$ & $b$ & $c$ & $d$ & $y$ \\
\hline
0 & 0 & 0 & 0 & 0 \\
0 & 0 & 0 & 1 & 0 \\
0 & 0 & 1 & 0 & 0 \\
0 & 0 & 1 & 1 & 1 \\
0 & 1 & 0 & 0 & 0 \\
0 & 1 & 0 & 1 & 0 \\
0 & 1 & 1 & 0 & 0 \\
0 & 1 & 1 & 1 & 1 \\
1 & 0 & 0 & 0 & 0 \\
1 & 0 & 0 & 1 & 0 \\
1 & 0 & 1 & 0 & 0 \\
1 & 0 & 1 & 1 & 1 \\
1 & 1 & 0 & 0 & 1 \\
1 & 1 & 0 & 1 & 1 \\
1 & 1 & 1 & 0 & 1 \\
1 & 1 & 1 & 1 & 1 \\
\hline
\end{tabular}
\end{table}

\section{Последовательные схемы}

\subsection{Часть 4: D-триггер}

\begin{table}[h]
\centering
\caption{Таблица истинности для D-триггера}
\begin{tabular}{|c|c|c|c|}
\hline
$reset$ & $d$ & $q_{n+1}$ & Описание \\
\hline
1 & X & 0 & Асинхронный сброс \\
0 & 0 & 0 & Передача 0 \\
0 & 1 & 1 & Передача 1 \\
\hline
\end{tabular}
\end{table}

\subsection{Часть 5: JK-триггер}

\begin{table}[h]
\centering
\caption{Таблица истинности для JK-триггера}
\begin{tabular}{|c|c|c|c|c|}
\hline
$reset$ & $j$ & $k$ & $q_{n+1}$ & Описание \\
\hline
1 & X & X & 0 & Асинхронный сброс \\
0 & 0 & 0 & $q_n$ & Сохранение состояния \\
0 & 0 & 1 & 0 & Сброс \\
0 & 1 & 0 & 1 & Установка \\
0 & 1 & 1 & $\overline{q_n}$ & Инверсия \\
\hline
\end{tabular}
\end{table}

\subsection{Часть 6: Последовательная схема с комбинационной логикой}

\begin{table}[h]
\centering
\caption{Таблица истинности для части 6}
\begin{tabular}{|c|c|c|c|c|c|}
\hline
$a$ & $b$ & $state_{n+1}$ & $state_n$ & $y$ & Описание \\
\hline
0 & 0 & 0 & 0 & 0 & Состояние 0, выход 0 \\
0 & 0 & 0 & 1 & 0 & Состояние 1, выход 0 \\
0 & 1 & 1 & 0 & 1 & Переход в состояние 1 \\
0 & 1 & 1 & 1 & 1 & Остаемся в состоянии 1 \\
1 & 0 & 1 & 0 & 0 & Переход в состояние 1 \\
1 & 0 & 1 & 1 & 1 & Остаемся в состоянии 1 \\
1 & 1 & 0 & 0 & 1 & Переход в состояние 0 \\
1 & 1 & 0 & 1 & 1 & Переход в состояние 0 \\
\hline
\end{tabular}
\end{table}

\textbf{Примечание:} $state_{n+1} = a \oplus b$, $y = state \cdot a + b$

\section{FSM модули}

\subsection{Счетчик 0-999}

\begin{table}[h]
\centering
\caption{Логика работы счетчика}
\begin{tabular}{|c|c|c|c|}
\hline
$reset$ & $count$ & $count_{n+1}$ & Описание \\
\hline
1 & X & 0 & Сброс счетчика \\
0 & 0-998 & $count + 1$ & Инкремент \\
0 & 999 & 0 & Переполнение \\
\hline
\end{tabular}
\end{table}

\subsection{Сдвиговый регистр-счетчик}

\begin{table}[h]
\centering
\caption{Логика работы сдвигового регистра}
\begin{tabular}{|c|c|c|c|c|}
\hline
$shift\_ena$ & $count\_ena$ & $data\_in$ & $data\_out_{n+1}$ & Описание \\
\hline
1 & 0 & X & $\{data\_out[2:0], data\_in\}$ & Сдвиг влево \\
0 & 1 & X & $data\_out - 1$ & Обратный счет \\
0 & 0 & X & $data\_out$ & Сохранение \\
\hline
\end{tabular}
\end{table}

\subsection{Детектор паттерна 1101}

\begin{table}[h]
\centering
\caption{Состояния FSM детектора паттерна}
\begin{tabular}{|c|c|c|c|}
\hline
Состояние & Вход & Следующее состояние & $start\_shifting$ \\
\hline
S0 (IDLE) & 0 & S0 & 0 \\
S0 (IDLE) & 1 & S1 & 0 \\
S1 & 0 & S0 & 0 \\
S1 & 1 & S2 & 0 \\
S2 & 0 & S3 & 0 \\
S2 & 1 & S2 & 0 \\
S3 & 0 & S0 & 0 \\
S3 & 1 & S4 & 1 \\
S4 & 0 & S0 & 0 \\
S4 & 1 & S1 & 0 \\
\hline
\end{tabular}
\end{table}

\subsection{Продвинутый таймер}

\begin{table}[h]
\centering
\caption{Состояния FSM продвинутого таймера}
\begin{tabular}{|c|c|c|c|c|}
\hline
Состояние & Условие перехода & Следующее состояние & $counting$ & $done$ \\
\hline
IDLE & Паттерн 1101 найден & SHIFTING & 0 & 0 \\
SHIFTING & 4 бита сдвинуты & COUNTING & 0 & 0 \\
COUNTING & $timer\_count \geq target$ & DONE\_STATE & 1 & 0 \\
DONE\_STATE & $ack = 1$ & IDLE & 0 & 1 \\
\hline
\end{tabular}
\end{table}

\textbf{Примечание:} $target\_count = (delay\_value + 1) \times 1000$

\section{Заключение}

Все таблицы истинности соответствуют реализованным модулям и подтверждают корректность их работы. Комбинационные схемы реализуют простые логические функции, а последовательные схемы демонстрируют различные типы триггеров и конечных автоматов.

\end{document}
